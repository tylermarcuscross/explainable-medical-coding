% Options for packages loaded elsewhere
\PassOptionsToPackage{unicode}{hyperref}
\PassOptionsToPackage{hyphens}{url}
\PassOptionsToPackage{dvipsnames,svgnames,x11names}{xcolor}
%
\documentclass[
  authoryear,
  preprint]{elsarticle}

\usepackage{amsmath,amssymb}
\usepackage{iftex}
\ifPDFTeX
  \usepackage[T1]{fontenc}
  \usepackage[utf8]{inputenc}
  \usepackage{textcomp} % provide euro and other symbols
\else % if luatex or xetex
  \usepackage{unicode-math}
  \defaultfontfeatures{Scale=MatchLowercase}
  \defaultfontfeatures[\rmfamily]{Ligatures=TeX,Scale=1}
\fi
\usepackage{lmodern}
\ifPDFTeX\else  
    % xetex/luatex font selection
\fi
% Use upquote if available, for straight quotes in verbatim environments
\IfFileExists{upquote.sty}{\usepackage{upquote}}{}
\IfFileExists{microtype.sty}{% use microtype if available
  \usepackage[]{microtype}
  \UseMicrotypeSet[protrusion]{basicmath} % disable protrusion for tt fonts
}{}
\makeatletter
\@ifundefined{KOMAClassName}{% if non-KOMA class
  \IfFileExists{parskip.sty}{%
    \usepackage{parskip}
  }{% else
    \setlength{\parindent}{0pt}
    \setlength{\parskip}{6pt plus 2pt minus 1pt}}
}{% if KOMA class
  \KOMAoptions{parskip=half}}
\makeatother
\usepackage{xcolor}
\setlength{\emergencystretch}{3em} % prevent overfull lines
\setcounter{secnumdepth}{5}
% Make \paragraph and \subparagraph free-standing
\makeatletter
\ifx\paragraph\undefined\else
  \let\oldparagraph\paragraph
  \renewcommand{\paragraph}{
    \@ifstar
      \xxxParagraphStar
      \xxxParagraphNoStar
  }
  \newcommand{\xxxParagraphStar}[1]{\oldparagraph*{#1}\mbox{}}
  \newcommand{\xxxParagraphNoStar}[1]{\oldparagraph{#1}\mbox{}}
\fi
\ifx\subparagraph\undefined\else
  \let\oldsubparagraph\subparagraph
  \renewcommand{\subparagraph}{
    \@ifstar
      \xxxSubParagraphStar
      \xxxSubParagraphNoStar
  }
  \newcommand{\xxxSubParagraphStar}[1]{\oldsubparagraph*{#1}\mbox{}}
  \newcommand{\xxxSubParagraphNoStar}[1]{\oldsubparagraph{#1}\mbox{}}
\fi
\makeatother


\providecommand{\tightlist}{%
  \setlength{\itemsep}{0pt}\setlength{\parskip}{0pt}}\usepackage{longtable,booktabs,array}
\usepackage{calc} % for calculating minipage widths
% Correct order of tables after \paragraph or \subparagraph
\usepackage{etoolbox}
\makeatletter
\patchcmd\longtable{\par}{\if@noskipsec\mbox{}\fi\par}{}{}
\makeatother
% Allow footnotes in longtable head/foot
\IfFileExists{footnotehyper.sty}{\usepackage{footnotehyper}}{\usepackage{footnote}}
\makesavenoteenv{longtable}
\usepackage{graphicx}
\makeatletter
\newsavebox\pandoc@box
\newcommand*\pandocbounded[1]{% scales image to fit in text height/width
  \sbox\pandoc@box{#1}%
  \Gscale@div\@tempa{\textheight}{\dimexpr\ht\pandoc@box+\dp\pandoc@box\relax}%
  \Gscale@div\@tempb{\linewidth}{\wd\pandoc@box}%
  \ifdim\@tempb\p@<\@tempa\p@\let\@tempa\@tempb\fi% select the smaller of both
  \ifdim\@tempa\p@<\p@\scalebox{\@tempa}{\usebox\pandoc@box}%
  \else\usebox{\pandoc@box}%
  \fi%
}
% Set default figure placement to htbp
\def\fps@figure{htbp}
\makeatother

\makeatletter
\@ifpackageloaded{caption}{}{\usepackage{caption}}
\AtBeginDocument{%
\ifdefined\contentsname
  \renewcommand*\contentsname{Table of contents}
\else
  \newcommand\contentsname{Table of contents}
\fi
\ifdefined\listfigurename
  \renewcommand*\listfigurename{List of Figures}
\else
  \newcommand\listfigurename{List of Figures}
\fi
\ifdefined\listtablename
  \renewcommand*\listtablename{List of Tables}
\else
  \newcommand\listtablename{List of Tables}
\fi
\ifdefined\figurename
  \renewcommand*\figurename{Figure}
\else
  \newcommand\figurename{Figure}
\fi
\ifdefined\tablename
  \renewcommand*\tablename{Table}
\else
  \newcommand\tablename{Table}
\fi
}
\@ifpackageloaded{float}{}{\usepackage{float}}
\floatstyle{ruled}
\@ifundefined{c@chapter}{\newfloat{codelisting}{h}{lop}}{\newfloat{codelisting}{h}{lop}[chapter]}
\floatname{codelisting}{Listing}
\newcommand*\listoflistings{\listof{codelisting}{List of Listings}}
\makeatother
\makeatletter
\makeatother
\makeatletter
\@ifpackageloaded{caption}{}{\usepackage{caption}}
\@ifpackageloaded{subcaption}{}{\usepackage{subcaption}}
\makeatother
\journal{Journal Name}

\usepackage[]{natbib}
\bibliographystyle{elsarticle-harv}
\usepackage{bookmark}

\IfFileExists{xurl.sty}{\usepackage{xurl}}{} % add URL line breaks if available
\urlstyle{same} % disable monospaced font for URLs
\hypersetup{
  pdftitle={CLEM-ICD - Clinical Language Encoding with ModernBERT},
  pdfauthor={Tyler Cross},
  pdfkeywords={clinical natural language processing, medical coding
automation, transformer models, multi-label classification, long-context
language models, ModernBERT, MIMIC-IV, ICD-10, healthcare
informatics, biomedical text classification},
  colorlinks=true,
  linkcolor={blue},
  filecolor={Maroon},
  citecolor={Blue},
  urlcolor={Blue},
  pdfcreator={LaTeX via pandoc}}


\setlength{\parindent}{6pt}
\begin{document}

\begin{frontmatter}
\title{CLEM-ICD - Clinical Language Encoding with
ModernBERT \\\large{Using ModernBERT to Improve Automated ICD-10
Classification} }
\author[1]{Tyler Cross%
\corref{cor1}%
\fnref{fn1}}
 \ead{tyler.cross@berkeley.com} 

\affiliation[1]{organization={University of California, Berkeley, School
of Information},addressline={102 South
Hall},city={Berkeley},postcode={94720},postcodesep={}}

\cortext[cor1]{Corresponding author}
\fntext[fn1]{Final project submission for the UC Berkeley Master of
Information and Data Science program, DATASCI 266: Natural Language
Processing with Deep Learning}
        
\begin{abstract}
Medical coding---the assignment of standardized ICD-10 codes to clinical
documentation---remains a labor-intensive process requiring expert
manual review of extensive narratives against 150,000+ potential
classifications. We present CLEM-ICD, an extension of the PLM-ICD
framework that replaces RoBERTa with ModernBERT to leverage its
8192-token context window for improved automated medical coding. Our
architectural enhancement preserves the multi-label classification
paradigm while significantly expanding the model's capacity to process
lengthy clinical text, capturing distant dependencies crucial for
accurate code assignment. Experiments on the MIMIC-IV dataset
demonstrate that CLEM-ICD achieves superior performance compared to
previous approaches, as measured by precision, recall, and F1-score.
This work addresses a critical bottleneck in healthcare administration
through advanced NLP techniques, offering a scalable solution for
reducing the cognitive burden of medical coding while maintaining
diagnostic accuracy.
\end{abstract}





\begin{keyword}
    clinical natural language processing \sep medical coding
automation \sep transformer models \sep multi-label
classification \sep long-context language
models \sep ModernBERT \sep MIMIC-IV \sep ICD-10 \sep healthcare
informatics \sep 
    biomedical text classification
\end{keyword}
\end{frontmatter}
    

Medical coding is a critical yet complex process in healthcare
administration. The translation of clinical documentation into
standardized ICD-10 codes is essential for healthcare reimbursement,
clinical research, and public health statistics. Currently, healthcare
providers rely on dedicated medical coders who manually review extensive
clinical notes to assign appropriate codes from over 150,000 possible
diagnostic and procedural codes, creating significant administrative
overhead \citep{tseng2018administrative}. Non-doctor workers outnumber
doctors in healthcare roughly 16 to 1 \citep{kocher2011rethinking}, with
medical coding representing a significant portion of this burden.

\section{Background}\label{background}

\citet{edin2024explainable} recently demonstrated success in medical
coding using a RoBERTa-based approach. Their work showed promising
results but left room for improvement in both performance and contextual
understanding. Other notable contributions in this field include
{[}additional related work to be filled in{]}.

ModernBERT \citep{warner2024modernbert} represents an advancement over
previous BERT-like models, offering an enhanced context window of 8192
tokens compared to the 512 tokens in traditional models. This increased
context capacity is particularly valuable for medical documents, which
tend to be lengthy and contain important information distributed
throughout the text.

\subsection{Using CSL}\label{using-csl}

\section{Methods}\label{methods}

This project utilizes the MIMIC-IV dataset, a large, freely available
database comprising de-identified health data associated with hospital
stays. The dataset includes detailed clinical notes, diagnostic codes,
procedural information, and other health-related data from real hospital
encounters.

\subsection{System Architecture}\label{system-architecture}

Our approach replaces the RoBERTa model used in previous work
\citep{edin2024explainable} with ModernBERT
\citep{warner2024modernbert}, leveraging its enhanced context window to
better process lengthy clinical notes. We hypothesize that this will
lead to improved code prediction by enabling the model to capture more
comprehensive context from medical documents.

\subsection{Implementation Details}\label{implementation-details}

{[}To be completed with specific implementation details{]}

\subsection{Evaluation Approach}\label{evaluation-approach}

We evaluate our system using standard performance metrics: precision,
recall, F1-score, and accuracy compared to gold-standard human coding,
following the evaluation protocol established in previous work.

\section{Results and Discussion}\label{results-and-discussion}

{[}This section will be completed after obtaining experimental
results{]}


\renewcommand\refname{References}
  \bibliography{bibliography.bib}



\end{document}
